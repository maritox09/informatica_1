\documentclass[]{article}
\usepackage[margin=0.5in]{geometry}
\begin{document}
\begin{center}
\Huge
Hoja de trabajo $\# 3$\\~\\
\end{center}
\Large
Ejercicio $\# 1$\\~\\
\normalsize
Realzar $[s(s(s(0)))] \oplus [s(s(s(s(0))))]$ paso por paso:\\~\\
\begin{center}
Empezamos con $[s(s(s(0)))] \oplus [s(s(s(s(0))))]$\\~\\
$\rightarrow s[s(s(0) \oplus s(s(s(s(0)))))]$\\~\\
$\rightarrow s[s[s(0) \oplus s(s(s(s(0))))]]$\\~\\
$\rightarrow s[s[s[0 \oplus s(s(s(s(0))))]]]$\\~\\
$\rightarrow s[s[s[s(s(s(s(0))))]]]$\\~\\
$\rightarrow s(s(s(s(s(s(s(0)))))))$\\~\\
\end{center}
\Large 
Ejercicio $\# 2$\\~\\
\normalsize
Definicion de multiplicacion para numeros unarios $(\otimes)$:\\~\\
\[
        n\otimes m = \left\{
        \begin{array}{l l}
            0 & \mbox{si } n=0 \\
            0 & \mbox{si } m=0 \\
            n \oplus (n \otimes i) & \mbox{si } m=s(i)\\
        \end{array}
        \right.
\]
\Large 
Ejercicio $\# 3$\\~\\
\normalsize
Verificacion de la definicion de la multiplicacion del problema $\# 2$:\\~\\
$s(s(s(0))) \otimes 0$:\\
\begin{center}
$s(s(s(0))) \otimes 0 = 0$ Por la segunda propiedad: $n \otimes m = 0 $ si $ m = 0$\\~\\
\end{center}
$s(s(s(0))) \otimes s(0)$:\\
\begin{center}
$s(s(s(0))) \otimes s(0)$\\~\\
$\rightarrow s(s(s(0))) \oplus [s(s(s(0))) \otimes 0]$\\~\\
Por la definicion sabemos que el resultado del lado derecho de la \emph{suma} es $0$\\~\\
$\rightarrow s(s(s(0))) \oplus 0$\\~\\
$\rightarrow s(s(s(0)))$
\end{center}
\newpage
$s(s(s(0))) \otimes s(s(0))$:\\
\begin{center}
$s(s(s(0))) \otimes s(s(0))$\\~\\
$\rightarrow s(s(s(0))) \oplus [s(s(s(0))) \oplus s(0)]$\\~\\
$\rightarrow s(s(s(0))) \oplus [s(s(s(0))) \oplus [s(s(s(0))) \otimes 0]]$\\~\\
$\rightarrow s(s(s(0))) \oplus [s(s(s(0))) \oplus 0]$\\~\\
$\rightarrow s(s(s(0))) \oplus s(s(s(0))$\\~\\
$\rightarrow s[s(s(0)) \oplus s(s(s(0)))]$\\~\\
$\rightarrow s[s[s(0) \oplus s(s(s(0)]]$\\~\\
$\rightarrow s[s[s[0 \oplus s(s(s(0))]]]$\\~\\
$\rightarrow s[s[s[s(s(s(0)))]]]$\\~\\
$\rightarrow s(s(s(s(s(s(0))))))$\\~\\
\end{center}
\Large
Ejercicio $\# 4$\\~\\
\normalsize
$a \oplus s(s(0)) = s(s(a))$:\\
\begin{center}
Tomamos la parte derecha de la igualdad y la modificamos de la siguiente forma:\\~\\
$\rightarrow a \oplus s(s(0))$\\~\\
$\rightarrow s(s(0)) \oplus a$\\~\\
$\rightarrow s[s(0)+a]$\\~\\
$\rightarrow s[s[0+a]]$\\~\\
Por definicion de la suma sabemos que $0+a=a$\\~\\
$\rightarrow s[s(a)]$\\~\\
$\rightarrow s(s(a)) = s(s(a))$\\~\\
\end{center}
\newpage
$a \otimes b = b \otimes a$:\\
\begin{center}
No pude :'(
\end{center}
$a \otimes (b \otimes c)=(a\otimes b)\otimes c$\\
\begin{center}
Esta tampoco :'(
\end{center}
$(a\oplus b)\otimes c = (a\otimes c) \oplus (b \otimes c)$
\begin{center}
Esta menos :'(
\end{center}
\end{document}