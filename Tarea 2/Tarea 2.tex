\documentclass[]{article}
\usepackage{amssymb}
\usepackage[utf8]{inputenc}
\usepackage[margin=0.5in]{geometry}
\begin{document}
\begin{center}
\Huge
Hoja de trabajo \# 2\\
\end{center}
\Large
Ejercicio \# 1\\
\normalsize
Demostración por inducción:\\
\begin{center}
$\forall n.n$ $n^3 \geq  n^2$ donde  $ n \in \mathbb{N}$\\~\\
Asumiendo que la expresión anterior es verdadera podemos inducir lo siguiente:\\~\\
$n^3 \geq  n^2$\\~\\
$\Longrightarrow \frac{n^3}{n^2} \geq  1$\\~\\
$\Longrightarrow n^{3-2} \geq 1$\\~\\
$\Longrightarrow n^1 \geq 1$\\~\\
$\Longrightarrow n \geq 1$\\~\\
Con esto se demuestra que el enunciado $n^3 \geq  n^2$ es verdadero para $\mathbb{N}$\\~\\
\end{center}
\Large
Ejercicio \# 2\\
\normalsize
Demostración por inducción:\\
\begin{center}
$\forall n.$ $(1+x)^n \geq nx$ Donde $n \in \mathbb{N}, x \in \mathbb{Q} $ y $ x \geq -1$\\~\\
Asumiendo que la expresión anterior es verdadera y $n = 1$ podemos inducir lo siguiente:\\
Usando el \emph{consejo} el lado izquierdo sera $nx +1$\\~\\
$(1+x)^n \geq 1+nx$\\~\\
$\Longrightarrow (1+x)^1 \geq 1+1x$\\~\\
$\Longrightarrow 1+x \geq 1+x$\\~\\
Con esto queda demostrado que la desigualdad es verdadera para $n=1$\\~\\
Ahora se utiliza el método de inducción suponiendo que $\forall n.$ $(1+x)^n \geq nx$ si $x \geq -1$\\
Sabiendo la condición $x \geq -1$ entonces $x+1 \geq 0$ así que se puede alterar ambos lados de la desigualdad sin cambiar la dirección de la misma.\\~\\
$(1+x)^{n+1} \geq (1+nx)(1+x)$\\~\\
$\Longrightarrow (1+x)^{n+1} \geq 1+x(n+1)+nx^2$\\~\\
Sabiendo que $n \in \mathbb{N}$ y $x^2 \geq 0$ entonces $nx^2 \geq 0$\\~\\
Esto verifica que $(1+x)^{n+1} \geq 1+x(n+1)$ de forma que $(1+x)^n \geq 1+nx$ también es aplicable con $n+1$\\~\\~\\~\\~\\
\end{center}
\small
Jose Mario Yon Cordon
\end{document}